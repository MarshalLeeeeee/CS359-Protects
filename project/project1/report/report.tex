\documentclass[12pt,a4paper]{article}
\usepackage{ctex}
\usepackage{amsmath,amscd,amsbsy,amssymb,latexsym,url,bm,amsthm}
\usepackage{epsfig,graphicx,subfigure}
\usepackage{enumitem,balance}
\usepackage{wrapfig}
\usepackage{mathrsfs, euscript}
\usepackage[usenames]{xcolor}
\usepackage{hyperref}
\usepackage{setspace}
\usepackage[vlined,ruled,commentsnumbered,linesnumbered]{algorithm2e}

\newtheorem{theorem}{Theorem}[section]
\newtheorem{lemma}[theorem]{Lemma}
\newtheorem{proposition}[theorem]{Proposition}
\newtheorem{corollary}[theorem]{Corollary}
\newtheorem{exercise}{Exercise}[section]
\newtheorem*{solution}{Solution}
\theoremstyle{definition}


\numberwithin{equation}{section}
\numberwithin{figure}{section}

\renewcommand{\thefootnote}{\fnsymbol{footnote}}

\newcommand{\postscript}[2]
 {\setlength{\epsfxsize}{#2\hsize}
  \centerline{\epsfbox{#1}}}

\renewcommand{\baselinestretch}{1.0}

\setlength{\oddsidemargin}{-0.365in}
\setlength{\evensidemargin}{-0.365in}
\setlength{\topmargin}{-0.3in}
\setlength{\headheight}{0in}
\setlength{\headsep}{0in}
\setlength{\textheight}{10.1in}
\setlength{\textwidth}{7in}
\makeatletter \renewenvironment{proof}[1][Proof] {\par\pushQED{\qed}\normalfont\topsep6\p@\@plus6\p@\relax\trivlist\item[\hskip\labelsep\bfseries#1\@addpunct{.}]\ignorespaces}{\popQED\endtrivlist\@endpefalse} \makeatother
\makeatletter
\renewenvironment{solution}[1][Solution] {\par\pushQED{\qed}\normalfont\topsep6\p@\@plus6\p@\relax\trivlist\item[\hskip\labelsep\bfseries#1\@addpunct{.}]\ignorespaces}{\popQED\endtrivlist\@endpefalse} \makeatother


%============================BEGIN OF THE DOCUMENT====================================
\begin{document}
\begin{spacing}{1.5}
%==========================BEGIN OF THE TITLE===================================
\begin{center}
\begin{spacing}{3.0}
\begin{tabular}{c c}
\multicolumn{2}{c}{\Huge{Report for Project-1 \hspace{2cm}}} \\
\hline
\large{Author: Li Minchao} & \large {ID: 515030910361}\\
\end{tabular}
\end{spacing}
\end{center}
%==========================END OF THE TITLE=====================================
%=========================BEGIN OF THE REPORT===================================
%==================BEGIN OF SECTION 1======================================
\section{Synopsis of the project}\par
\subsection{Demand of the project}
To achieve some bit-level function using C grammar.\par
The limitation is to complete each function skeleton using only straightline code for the integer puzzles and a limited number of C arithmetic and logical operators.\par
The operator set for each function is given, and the max op is anticipated.\par
\subsection{Purpose of the project}
To become more familiar with bit-level representations of integers and floating numbers.\par
To work my own way through working it and make the algorithm as simple as possible.\par
%===================END OF SECTION 1=======================================
%==================BEGIN OF SECTION 2======================================
\section{Environment of the project}
OS:Linux Ubuntu Kylin 16.04\par
Compile under gcc v5.4.0\par
%===================END OF SECTION 2=======================================
%==================BEGIN OF SECTION 3======================================
\section{Implementation process of the project}
I will illustrate how I managed to implement the function and conquer the barriers in this section.

\subsection{bitAnd}
\textbf{Introduction: }receive two arguments $\left(x, y\right)$, and return x $\&$ y using only $\sim$ and $\mid$\par
\textbf{Solution: }\par
Mainly use the De Morgan's law, $\overline{A \cup B} = \overline{A} \cap \overline{B}$.\par
Thus, in this case, we have the equation, $\overline{\overline{A} \cup \overline{B}} = A \cap B$ to solve this problem.\par

\subsection{getByte}
\textbf{Introduction: }receive two arguments $\left(x, n\right)$, and return the extract byte n from word x.\par
\textbf{Solution: }\par
Since one byte has 8 bits, the shift amount (I name this para as movIndex in the function) should be 8 $\cdot$ n.\par
Thus, we get the equation, $movIndex = n \ll 3$.\par
And the answer is $\left(x \gg movIndex\right)\,\&\,\textrm{0xff}$.\par

\subsection{logicalShift}
\textbf{Introduction: }receive two arguments $\left(x, n\right)$, and shift x to the right by n, using a logical shift\par
\textbf{Solution: }\par
I declare a parament base = $1 \ll 31$ = 0x80000000.\par
Since base is marked as signed integer, during the process of left shift, 1 is extended.\par
By using $\sim\left(base \gg \left(n-1\right)\right)$, we get $00\cdots0\underbrace{11\cdots1}_{32-n\;1s}$, which serve as the mask.\par
However, - operation is not allowed. To achieve this operation, we can implement $+\left(\sim 1 + 1\right)$.\par
While in this case, if we directly use n + 0xffffffff, we will get unexpected case when n is 0.\par
Since the low bit is always 0, we can left shift n, and right shift 1. So the mask is always what I expect.\par
Taking the advantage of the mask, no matter whether the high bit is extended by 0 or 1, we can always have the right answer.\par

\subsection{bitCount}
\textbf{Introduction: }receive one argument x, and returns count of number of 1's in word.\par
\textbf{Solution: }\par
This is quite difficult for me, and probably the hardest one in this project.\par
The ceiling of operation number is too limited, and I cannot make a simple solution.\par
I bing-ed the relative solution, and there I found the solution based on divide and conquer methodology.\par
I finally accomplish this function by the hint and I will explain how it works above.\par
The core is that x saves the answer of partial bits in its corresponding part.\par
First, we have this assignment, $x = \left(x\,\&\, 0x55555555\right) + \left(\left(x \gg 1\right)\,\&\, 0x55555555\right)$, which counts the number of 1's of some 2  consecutive bits.\par
In the equation above, $x\,\&\,0x55555555$ and $\left(x \gg 1\right)\,\&\, 0x55555555$ represents the bits only in odd or even position.\par
Take 0x1110 as an example, we add 0x0100 and 0x0101 (which is the result of $x\,\&\,0x55555555$ and $\left(x \gg 1\right)\,\&\, 0x55555555$), and get 0x1001 as the result, which 10 stands for 2, is the number of 1's in 11(the 2 and 3 bit) and 01 stands for 1, is the number of 1's in 10(the 1 and 0 bit).\par
Every assignment in the function is to add two consecutive blocks which store answer of partial bits. The add operation is between either of 1-bit operands or 2-bit operands or 4-bit operands or 8-bit operands or 16-bit operands.\par
We can view the initial x as one bit store the answer of 1's in one bit, and generally x use n bits to store the answer of 1's in n bits. Finally, we use 32 bits to store the answer of the 1's of the whole initial x.\par
However, we cannot use those big int straightly.\par
To overcome this, I declare 5 variables and got the value using $\ll$ and $\mid$.\par
For example:\par
mask1 = $\left(0x55 \ll 8\right)\,|\,0x55$;\par
0x55555555 = $\left(mask1 \ll 16\right)\,|\,mask1$;\par

\subsection{bang}
\textbf{Introduction: }receive one argument x, and return !x without using !.\par
\textbf{Solution: }\par
If there is any 1 in any bit, we return 0.\par
So, what I do is to calculate $b_0 \cup b_1 \cup \cdots \cup b_{31}$.\par
First, I do this assignment $x = x \mid \left(x \gg 16\right)$, which record information of 32 bits into the low 16 bits.\par
If there is any high bit in either low 16 bits or high 16 bits, the low 16 bits will have high bit.\par
Likely, I do the similar assignments to condense information into less bits.\par
Finally, all information is recorded in the $b_0$.\par
Return 0, if $b_0$ is 1, and return 1, if $b_0$ is 0.\par

\subsection{tmin}
\textbf{Introduction: }receive no arguments, and return minimum two's complement integer.\par
\textbf{Solution: }\par
Just return $1 \ll 31$.\par

\subsection{fitsBits}
\textbf{Introduction: }receive two arguments $\left(x, n\right)$, and return 1 if x can be represented as an n-bit, two's complement integer.\par
\textbf{Solution: }\par
If I define the bit sequence is $\left(b_{31},b_{30},\cdots,b_1,b_0\right)$, the key of this function is to see if the $\left(b_{31},b_{30},\cdots,b_{n-1}\right)$ is the sequence made of same number.\par
Since x is a signed integer, sign extended will be exerted when right shifted. So I left shift (n-1) bits, and if all bits are 0 or 1, return 1.\par
Well, how can I implement n-1. One way is to add 0xffffffff, however, such integer is not allowed in this project. So, I use a trick to make 0xffffffff by implementing $\left(1 \ll 31\right) \gg 31$.\par
To determine this, I just need to add $b_{n-1}$ to the number. If the result is 0, the function will return 1.\par

\subsection{divpwr2}
\textbf{Introduction: }receive two arguments $\left(x, n\right)$, and return $x/\left(2^{n}\right), for\;0 \leq n \leq 30$.\par
\textbf{Solution: }\par
In the case of $x \geq 0$, it is quite easy, we just need to left shift n.\par
In the case of $x < 0$, it is complicated. We have to consider when and when not to add 1 to the result.\par
Use $sgn = \left(x \gg 31\right)\;\&\;1$ to record the sign bit.\par
If x can be divided by $2^{n}$, we don't have to add 1. I get the low bit of x by $x\,\&\left(\sim x + 1\right)$. And if we left shift low bit by n is not zero, it means x can be divided by $2^{n}$.\par
We left shift the low bit by n bits, if the result is not zero, it means x can be divided by $2^{n}$.\par
Use $sgn2 = !\left(lowbit \gg n\right)$ to mark this situation.\par
Return $\left(x \gg n\right) + \left(sgn\;\&\;sgn2\right)$

\subsection{negate}
\textbf{Introduction: }receive one argument x, and return -x.\par
\textbf{Solution: }\par
According to the 2'compliment rule, simply return $\sim x + 1$.\par

\subsection{isPositive}
\textbf{Introduction: }receive one argument x, and return 1 if $x > 0$, otherwise, return 0.\par
\textbf{Solution: }\par
For a signed integer, the highest bit indicates if it's positive.\par
If it's 1, it's negative, otherwise, it's positive.\par
Use $sgn = \left(x \gg 31\right)\;\&\;1$ to record the bit.\par
However, we should return 0, if x = 0.\par
So, we should return $\left(!sgn\right)\;\&\; !\left(!x\right)$

\subsection{isLessOrEqual}
\textbf{Introduction: }receive two arguments $\left(x, y\right)$, and return 1 if $x \leq y$, otherwise, return 0.\par
\textbf{Solution: }\par
I notate the sign of x and y as x\underline{ }sgn and y\underline{ }sgn. I notate not\underline{ }equal = x\underline{ }sgn $\land$ y\underline{ }sgn.\par
If x and y have different sign, not\underline{ }equal = 1. And the answer should be x\underline{ }sgn.\par
If x and y have same sign, not\underline{ }equal = 0. And the answer should depend whether $y + \left(\sim x + 1\right) \geq 0$.\par
So we only need to check whether the highest bit of $y + \left(\sim x + 1\right)$ is 0.\par
Finally, use the structure of $\left(\textrm{not\underline{ }equal}\;\&\;\cdots\right) \mid \left(\textrm{!not\underline{ }equal}\;\&\;\cdots\right)$ to do the branch.\par

\subsection{ilog2}
\textbf{Introduction: }receive one argument x, and return floor(log base 2 of x), where $x > 0$.\par
\textbf{Solution: }\par
The key of this function is to return the highest index such that $b_{index} = 1$. Index ranges from 0 to 31.\par
Since $x > 0$, 0 is extended if x is left shifted.\par
If $x \gg 16 = 0$, it means we don't have 1 at the high 16 bits. Otherwise, the answer is at least 16 big. We use $sgn = !\left(!\left(x \gg 16\right)\right)$ to mark if we have 1 at high bits. And we use $cnt = sgn \ll 4$ to accumulate the answer.\par
After these operation, we left shift x, $x = x \gg cnt$. If we don't have 1 at high bits, we then only focus on the low bits ranging from 0 to 15. If we have  1 at high bits, we then focus on the bits ranging from 16 to 31, and ignore the low bits, since we already know high bit is at 16 to 31.\par
Do the similar operation, and change the constant from 16 to 8 and 4 and 2 and 1. And we can traverse the number bits.\par
Finally we accumulate those cnt, and return the sum.\par

\subsection{float\underline{ }neg}
\textbf{Introduction: }receive one argument uf, and return bit-level equivalent of expression -f for floating point argument f. Both the argument and result are passed as unsigned int's, but they are to be interpreted as the bit-level representations of single-precision floating point values. When argument is NaN, return argument.\par
\textbf{Solution: }\par
Floating number is mainly divided into three parts: s, e, m.\par
s, is the $b_{31}$, represents the sign of the number.\par
e, ranges from $b_{23}\;to\;b_{30}$, represents the power of 2.\par
m, ranges from $b_0\;to\;b_{22}$, represents the base.\par
To get the s, e, m part, we can shift the number and \& the corresponding mask.\par
If uf doesn't represent NAN, which means e part is 0xff and m part is not 0, simply return uf.\par
If not, we simply change the sign bit.

\subsection{float\underline{ }i2f}
\textbf{Introduction: }receive one argument x, and return bit-level equivalent of expression (float) x. Result is returned as unsigned int, but it is to be interpreted as the bit-level representation of a single-precision floating point values.\par
\textbf{Solution: }\par
Do branch case if x is 0 or 0x80000000.\par
Detect and save the sign of x, if x is negative, we assign $x = -x$.\par
Then we should get the index of the highest bit of 1, the implement of which is similar with ilog2.\par
If $index > 23$, we should also consider the deviation part which determines whether we take the floor or the ceiling to minimize the deviation. If the deviation is the same, we select the number which is even.\par
The value of index also serve us to calculate e part. We just simply add index to the offset 127 and assign the result to e.\par
Everything logical can be accomplished by branching. Finally using the mask skills to combine s, e, m part.\par
However, I fail to accomplish this function within the max op.I cannot carry out any way to limit total operations into 30.\par

\subsection{float\underline{ }twice}
\textbf{Introduction: }receive one argument uf, and return bit-level equivalent of expression $2\cdot f$ for floating point argument f. Both the argument and result are passed as unsigned int's, but they are to be interpreted as the bit-level representation of single-precision floating point values. When argument is NaN, return argument.\par
\textbf{Solution: }\par
Regularly, I calculate the value of s, e, m part using shift operation and mask skill.\par
If e part is zero, it means we cannot shift e part for e is always 0, and we should left shift the m by 1.\par
Else if e part is 0xff, we just return uf. Since no matter it's infinite or NAN, uf itself is the answer of the function.\par
Else, we simply add 1 to e part, and refresh e part to the initial uf.\par
%===================END OF SECTION 3=======================================
%==================BEGIN OF SECTION 4======================================
\section{Review of the project}
This project makes me to think about many details of those familiar operations and realize the implement or accomplishment of those seemingly regular functions is not that easy. Never will I know how it works until I try it myself, since these functions are already packaged. I can feel the efficiency of those standard implementation of the function compared with my naive way of implementation.\par
Also, I come to know how the floating number is actually stored which is only a vague picture in my mind before. However, comprehension is one thing and practise is another. there are more things to be taken into consideration since the variable state of floating number and the existing deviation compared with integers.\par
Finally, I appreciate it much for some online documents and blogs which solve many of my confusion.\par
%===================END OF SECTION 4=======================================
%==================BEGIN OF SECTION 5======================================
\section{Reference}
http://www.cnblogs.com/graphics/archive/2010/06/21/1752421.html\par
https://wenku.baidu.com/view/f8fc95fad1f34693daef3e67.html\par
http://blog.csdn.net/zhzhanp/article/details/6339883\par
%===================END OF SECTION 5=======================================
%==========================END OF THE REPORT====================================
\end{spacing}
\end{document}
%============================END OF THE DOCUMENT======================================
